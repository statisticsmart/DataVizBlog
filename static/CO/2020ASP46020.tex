%%%%%%%%%%%%%%%%%%%%%%%%%%%%%%%%%%%%%%%%%%%%%%%%%%%%%
%%%%%	 		Course Outline- Template		%%%%%
%%%%%		  Prepaired by Dr. R.M. Silva		%%%%%
%%%%%			Department of Statistics		%%%%%
%%%%%		  Faculty of Applied Sciences		%%%%%
%%%%%		University of Sri Jayewardenepura   %%%%%
%%%%%				Dec. 09th, 2017				%%%%%
%%%%%%%%%%%%%%%%%%%%%%%%%%%%%%%%%%%%%%%%%%%%%%%%%%%%%


% Please place the SJPLogo in the same folder that you are going to run this Latex file

\documentclass[a4paper,12pt]{article}
\usepackage{graphicx}
\usepackage{enumitem}
\usepackage{setspace}
\usepackage{multicol} 
\usepackage{color}
\usepackage[
top    = 2.00cm,
bottom = 2.00cm,
left   = 2.00cm,
right  = 2.00cm]{geometry}
\usepackage{hyperref}
\hypersetup{pdfstartview={XYZ null null 1.00}}

\begin{document}

\begin{figure}[ht]
	\begin{center}
		\includegraphics[angle=0,scale=0.05]{SJPLogo.jpg}
	\end{center}
\end{figure}

\vspace{-1cm}

\begin{center}
	University of Sri Jayewardenepura\\
	Faculty of Applied Sciences \\
	Department of Statistics
\end{center}


\noindent\rule{17cm}{0.4pt} 	%This will create a horizontal line

\begin{multicols}{3}
	\noindent\textbf{Batch: 2015/2016}
	
	\columnbreak
	\noindent\textbf{Year: 2020 }
	
	\columnbreak
	\noindent\textbf{Semester: First Semester}
	
\end{multicols}

\noindent\rule{17cm}{0.4pt}	% horizontal line

\vspace{0.5cm}
\noindent\textbf{Course Unit: ASP 460 2.0 Special Topics in Statistics}\\

\noindent\textbf{Type of the course unit: Optional}\\

\noindent\textbf{Pre-Requisites:}\\
\noindent\textit{{STA 124 1.5 Data Analysis I,  STA 226 1.5 Data Analysis II, STA 326 2.0 Programming and Data Analysis with R}}\\

\noindent\textbf{Workload:}\\
\noindent\textit{{ Minimum total expected workload to achieve the learning outcomes for this unit is 100 hours per semester typically comprising a mixture of scheduled learning activities, independent study and 26 hours of lectures. Independent study may include associated readings, assessment and preparation for scheduled activities.}}\\

\noindent\textbf{Course Objectives:}
\begin{itemize}
	\setlength\itemsep{0.1mm}
	\item To introduce data visualization principles, theories and techniques.
	\item To introduce how to better understand your data, present findings and tell engaging data stories that clearly depict the points you want to make all through data graphics.
\end{itemize}
\noindent\textbf{Course Contents:}
\begin{enumerate}[label*=\arabic*.]
	\setlength\itemsep{0.1mm}
	\item Introduction to data visualization: History of data visualization, Design principles, Visualization design process
	\item Scientific design choices in data visualization: Static graphics, Choice of graphical form, Display options (scale, colour, sorting, annotation, positioning, etc)
	\item The grammar of graphics
	\item Higher-dimensional displays and special structures: Scatterplot matrices, Parallel coordinates, Mosaic plots, Small multiples and trellis displays
	\item Visualization of high-dimensional data
	\item Visualization of multivariate data, time series data and spatial data
	\item Linked data views for visual exploration
  \item Dashboards, interactive and animated displays
\end{enumerate}

\newpage

\noindent\textbf{Learning Outcomes:}
At the end of this course, the student should be able to:
\begin{itemize}
	\setlength\itemsep{0.1mm}
\item Define principles of good visualization design.
\item Explain design principles.
\item Identify appropriate data visualization techniques.
\item Create data graphics using the ggplot2 package.
\item Design and create data visualizations for your target audience and task.
\item Conduct exploratory data analysis using visualization.
\item Compare different data visualizations.
\item Craft visual presentations of data for effective communication.
\item Develop dynamic visualizations that allow others to interact with data.
\item Critique existing visualizations based on data visualization theory and principles.
\item Revise data visualizations using appropriate design principles.


\end{itemize}

\noindent\textbf{Method of Assessment:}
\begin{itemize}
	\setlength\itemsep{0.1mm}
	\item Continuous assessment: 40\%
	\item Final project: 60\%
\end{itemize}

\noindent\textbf{Recommended Readings:}
\begin{itemize}
	\setlength\itemsep{0.1mm}
	\item Title: Handbook of Data Visualization \\
		  Author(s): C. Chen, W. Hardle and A. Unwin \\
		  Publisher: Springer \\
	\item Title: R for Data Science \\
		  Author(s): Hadley Wickham and Garrett Grolemund \\
		  Publisher: O'REILLY \\
		  This book is available online for free. Visit \textcolor{blue}{https://r4ds.had.co.nz/}
\end{itemize}

\noindent\textbf{Lecturer in Charge:}

\begin{itemize}
	\setlength\itemsep{0.1mm}
	\item[] Name: Dr Thiyanga  Talagala
	\item[] Email: ttalagala@sjp.ac.lk
\end{itemize}


\end{document}
