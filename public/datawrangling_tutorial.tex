\PassOptionsToPackage{unicode=true}{hyperref} % options for packages loaded elsewhere
\PassOptionsToPackage{hyphens}{url}
%
\documentclass[]{article}
\usepackage{lmodern}
\usepackage{amssymb,amsmath}
\usepackage{ifxetex,ifluatex}
\usepackage{fixltx2e} % provides \textsubscript
\ifnum 0\ifxetex 1\fi\ifluatex 1\fi=0 % if pdftex
  \usepackage[T1]{fontenc}
  \usepackage[utf8]{inputenc}
  \usepackage{textcomp} % provides euro and other symbols
\else % if luatex or xelatex
  \usepackage{unicode-math}
  \defaultfontfeatures{Ligatures=TeX,Scale=MatchLowercase}
\fi
% use upquote if available, for straight quotes in verbatim environments
\IfFileExists{upquote.sty}{\usepackage{upquote}}{}
% use microtype if available
\IfFileExists{microtype.sty}{%
\usepackage[]{microtype}
\UseMicrotypeSet[protrusion]{basicmath} % disable protrusion for tt fonts
}{}
\IfFileExists{parskip.sty}{%
\usepackage{parskip}
}{% else
\setlength{\parindent}{0pt}
\setlength{\parskip}{6pt plus 2pt minus 1pt}
}
\usepackage{hyperref}
\hypersetup{
            pdftitle={ASP 460 2.0 Special topics in Statistics: Data Wrangling},
            pdfborder={0 0 0},
            breaklinks=true}
\urlstyle{same}  % don't use monospace font for urls
\usepackage[margin=1in]{geometry}
\usepackage{color}
\usepackage{fancyvrb}
\newcommand{\VerbBar}{|}
\newcommand{\VERB}{\Verb[commandchars=\\\{\}]}
\DefineVerbatimEnvironment{Highlighting}{Verbatim}{commandchars=\\\{\}}
% Add ',fontsize=\small' for more characters per line
\usepackage{framed}
\definecolor{shadecolor}{RGB}{248,248,248}
\newenvironment{Shaded}{\begin{snugshade}}{\end{snugshade}}
\newcommand{\AlertTok}[1]{\textcolor[rgb]{0.94,0.16,0.16}{#1}}
\newcommand{\AnnotationTok}[1]{\textcolor[rgb]{0.56,0.35,0.01}{\textbf{\textit{#1}}}}
\newcommand{\AttributeTok}[1]{\textcolor[rgb]{0.77,0.63,0.00}{#1}}
\newcommand{\BaseNTok}[1]{\textcolor[rgb]{0.00,0.00,0.81}{#1}}
\newcommand{\BuiltInTok}[1]{#1}
\newcommand{\CharTok}[1]{\textcolor[rgb]{0.31,0.60,0.02}{#1}}
\newcommand{\CommentTok}[1]{\textcolor[rgb]{0.56,0.35,0.01}{\textit{#1}}}
\newcommand{\CommentVarTok}[1]{\textcolor[rgb]{0.56,0.35,0.01}{\textbf{\textit{#1}}}}
\newcommand{\ConstantTok}[1]{\textcolor[rgb]{0.00,0.00,0.00}{#1}}
\newcommand{\ControlFlowTok}[1]{\textcolor[rgb]{0.13,0.29,0.53}{\textbf{#1}}}
\newcommand{\DataTypeTok}[1]{\textcolor[rgb]{0.13,0.29,0.53}{#1}}
\newcommand{\DecValTok}[1]{\textcolor[rgb]{0.00,0.00,0.81}{#1}}
\newcommand{\DocumentationTok}[1]{\textcolor[rgb]{0.56,0.35,0.01}{\textbf{\textit{#1}}}}
\newcommand{\ErrorTok}[1]{\textcolor[rgb]{0.64,0.00,0.00}{\textbf{#1}}}
\newcommand{\ExtensionTok}[1]{#1}
\newcommand{\FloatTok}[1]{\textcolor[rgb]{0.00,0.00,0.81}{#1}}
\newcommand{\FunctionTok}[1]{\textcolor[rgb]{0.00,0.00,0.00}{#1}}
\newcommand{\ImportTok}[1]{#1}
\newcommand{\InformationTok}[1]{\textcolor[rgb]{0.56,0.35,0.01}{\textbf{\textit{#1}}}}
\newcommand{\KeywordTok}[1]{\textcolor[rgb]{0.13,0.29,0.53}{\textbf{#1}}}
\newcommand{\NormalTok}[1]{#1}
\newcommand{\OperatorTok}[1]{\textcolor[rgb]{0.81,0.36,0.00}{\textbf{#1}}}
\newcommand{\OtherTok}[1]{\textcolor[rgb]{0.56,0.35,0.01}{#1}}
\newcommand{\PreprocessorTok}[1]{\textcolor[rgb]{0.56,0.35,0.01}{\textit{#1}}}
\newcommand{\RegionMarkerTok}[1]{#1}
\newcommand{\SpecialCharTok}[1]{\textcolor[rgb]{0.00,0.00,0.00}{#1}}
\newcommand{\SpecialStringTok}[1]{\textcolor[rgb]{0.31,0.60,0.02}{#1}}
\newcommand{\StringTok}[1]{\textcolor[rgb]{0.31,0.60,0.02}{#1}}
\newcommand{\VariableTok}[1]{\textcolor[rgb]{0.00,0.00,0.00}{#1}}
\newcommand{\VerbatimStringTok}[1]{\textcolor[rgb]{0.31,0.60,0.02}{#1}}
\newcommand{\WarningTok}[1]{\textcolor[rgb]{0.56,0.35,0.01}{\textbf{\textit{#1}}}}
\usepackage{graphicx,grffile}
\makeatletter
\def\maxwidth{\ifdim\Gin@nat@width>\linewidth\linewidth\else\Gin@nat@width\fi}
\def\maxheight{\ifdim\Gin@nat@height>\textheight\textheight\else\Gin@nat@height\fi}
\makeatother
% Scale images if necessary, so that they will not overflow the page
% margins by default, and it is still possible to overwrite the defaults
% using explicit options in \includegraphics[width, height, ...]{}
\setkeys{Gin}{width=\maxwidth,height=\maxheight,keepaspectratio}
\setlength{\emergencystretch}{3em}  % prevent overfull lines
\providecommand{\tightlist}{%
  \setlength{\itemsep}{0pt}\setlength{\parskip}{0pt}}
\setcounter{secnumdepth}{0}
% Redefines (sub)paragraphs to behave more like sections
\ifx\paragraph\undefined\else
\let\oldparagraph\paragraph
\renewcommand{\paragraph}[1]{\oldparagraph{#1}\mbox{}}
\fi
\ifx\subparagraph\undefined\else
\let\oldsubparagraph\subparagraph
\renewcommand{\subparagraph}[1]{\oldsubparagraph{#1}\mbox{}}
\fi

% set default figure placement to htbp
\makeatletter
\def\fps@figure{htbp}
\makeatother


\title{ASP 460 2.0 Special topics in Statistics: Data Wrangling}
\author{}
\date{\vspace{-2.5em}13/05/2020}

\begin{document}
\maketitle

\hypertarget{packages}{%
\section{Packages}\label{packages}}

\hypertarget{install-tidyverse-into-your-computer.}{%
\subsection{Install tidyverse into your
computer.}\label{install-tidyverse-into-your-computer.}}

\begin{Shaded}
\begin{Highlighting}[]
\KeywordTok{install.packages}\NormalTok{(}\StringTok{"tidyverse"}\NormalTok{)}
\end{Highlighting}
\end{Shaded}

\hypertarget{we-are-going-to-work-with-datasets-in-edawr.}{%
\subsection{\texorpdfstring{We are going to work with datasets in
\texttt{EDAWR}.}{We are going to work with datasets in EDAWR.}}\label{we-are-going-to-work-with-datasets-in-edawr.}}

Installation of EDAWR is bit different. Use the following command.

\textbf{Step 1}

\begin{Shaded}
\begin{Highlighting}[]
\KeywordTok{install.packages}\NormalTok{(}\StringTok{"devtools"}\NormalTok{)}
\end{Highlighting}
\end{Shaded}

\textbf{Step 2}

\begin{Shaded}
\begin{Highlighting}[]
\NormalTok{devtools}\OperatorTok{::}\KeywordTok{install_github}\NormalTok{(}\StringTok{"rstudio/EDAWR"}\NormalTok{)}
\end{Highlighting}
\end{Shaded}

\hypertarget{load-packages}{%
\subsection{Load packages}\label{load-packages}}

\begin{Shaded}
\begin{Highlighting}[]
\KeywordTok{library}\NormalTok{(tidyverse)}
\end{Highlighting}
\end{Shaded}

\begin{verbatim}
## -- Attaching packages ------------------ tidyverse 1.3.0 --
\end{verbatim}

\begin{verbatim}
## v ggplot2 3.3.0     v purrr   0.3.4
## v tibble  3.0.1     v dplyr   0.8.5
## v tidyr   1.0.3     v stringr 1.4.0
## v readr   1.3.1     v forcats 0.5.0
\end{verbatim}

\begin{verbatim}
## -- Conflicts --------------------- tidyverse_conflicts() --
## x dplyr::filter() masks stats::filter()
## x dplyr::lag()    masks stats::lag()
\end{verbatim}

\begin{Shaded}
\begin{Highlighting}[]
\KeywordTok{library}\NormalTok{(EDAWR) }\CommentTok{# load data}
\end{Highlighting}
\end{Shaded}

\begin{verbatim}
## 
## Attaching package: 'EDAWR'
\end{verbatim}

\begin{verbatim}
## The following object is masked from 'package:dplyr':
## 
##     storms
\end{verbatim}

\begin{verbatim}
## The following objects are masked from 'package:tidyr':
## 
##     population, who
\end{verbatim}

\hypertarget{pipe-operator-dplyr-and-tidyr}{%
\section{Pipe operator, dplyr, and
tidyr}\label{pipe-operator-dplyr-and-tidyr}}

\begin{itemize}
\item
  \texttt{dplyr} is a package for data wrangling, with several key verbs
  (functions).
\item
  \texttt{slice()} and \texttt{filter()}: subset rows based on numbers
  or conditions.
\item
  \texttt{select()} and \texttt{pull()}: select columns or a single
  column as a vector.
\item
  \texttt{arrange()}: order rows by one or multiple columns.
\item
  \texttt{rename()} and \texttt{mutate()}: rename or create columns.
\item
  \texttt{mutate\_at()}: apply a function to given columns.
\end{itemize}

\hypertarget{recall-pipe-operator}{%
\section{\texorpdfstring{Recall: Pipe operator
(\texttt{\%\textgreater{}\%})}{Recall: Pipe operator (\%\textgreater{}\%)}}\label{recall-pipe-operator}}

See the slides in STA 326 2.0

Link: :
\url{https://hellor.netlify.app/slides/l7_intro_tidyverse.html\#43}

\begin{Shaded}
\begin{Highlighting}[]
\NormalTok{iris }\OperatorTok\StringTok{ }
\StringTok{  }\KeywordTok{filter}\NormalTok{(Sepal.Length }\OperatorTok{>=}\StringTok{ }\DecValTok{7}\NormalTok{) }
\end{Highlighting}
\end{Shaded}

\begin{verbatim}
   Sepal.Length Sepal.Width Petal.Length Petal.Width    Species
1           7.0         3.2          4.7         1.4 versicolor
2           7.1         3.0          5.9         2.1  virginica
3           7.6         3.0          6.6         2.1  virginica
4           7.3         2.9          6.3         1.8  virginica
5           7.2         3.6          6.1         2.5  virginica
6           7.7         3.8          6.7         2.2  virginica
7           7.7         2.6          6.9         2.3  virginica
8           7.7         2.8          6.7         2.0  virginica
9           7.2         3.2          6.0         1.8  virginica
10          7.2         3.0          5.8         1.6  virginica
11          7.4         2.8          6.1         1.9  virginica
12          7.9         3.8          6.4         2.0  virginica
13          7.7         3.0          6.1         2.3  virginica
\end{verbatim}

\begin{Shaded}
\begin{Highlighting}[]
\NormalTok{iris }\OperatorTok\StringTok{ }
\StringTok{  }\KeywordTok{filter}\NormalTok{(Sepal.Length }\OperatorTok{>=}\StringTok{ }\DecValTok{7}\NormalTok{) }\OperatorTok\StringTok{ }
\StringTok{  }\KeywordTok{head}\NormalTok{(}\DecValTok{2}\NormalTok{)}
\end{Highlighting}
\end{Shaded}

\begin{verbatim}
  Sepal.Length Sepal.Width Petal.Length Petal.Width    Species
1          7.0         3.2          4.7         1.4 versicolor
2          7.1         3.0          5.9         2.1  virginica
\end{verbatim}

\hypertarget{tidyr-functions-they-are-also-called-tidy-r-verbs}{%
\subsection{\texorpdfstring{\texttt{tidyr} functions (They are also
called tidy R
verbs)}{tidyr functions (They are also called tidy R verbs)}}\label{tidyr-functions-they-are-also-called-tidy-r-verbs}}

Main verbs (functions) in \texttt{tidyr}:

\begin{itemize}
\item
  \texttt{pivot\_longer()}: makes datasets longer by increasing the
  number of rows and decreasing the number of columns.
\item
  \texttt{pivot\_wider()}: is the opposite of pivot\_longer() : it makes
  a dataset wider by increasing the number of columns and decreasing the
  number of rows.
\item
  \texttt{separate()}: splits a single column into multiple columns.
\item
  \texttt{unite()}: combines multiple columns into a single column.
\end{itemize}

\hypertarget{pivot_longer}{%
\section{\texorpdfstring{\texttt{pivot\_longer()}}{pivot\_longer()}}\label{pivot_longer}}

\begin{Shaded}
\begin{Highlighting}[]
\CommentTok{# EDAWR::cases means cases dataset in EDAWR}
\NormalTok{EDAWR}\OperatorTok{::}\NormalTok{cases }\OperatorTok\StringTok{ }
\StringTok{  }\KeywordTok{head}\NormalTok{(}\DecValTok{3}\NormalTok{)}
\end{Highlighting}
\end{Shaded}

\begin{verbatim}
##   country  2011  2012  2013
## 1      FR  7000  6900  7000
## 2      DE  5800  6000  6200
## 3      US 15000 14000 13000
\end{verbatim}

\begin{Shaded}
\begin{Highlighting}[]
\NormalTok{EDAWR}\OperatorTok{::}\NormalTok{cases }\OperatorTok\StringTok{ }
\StringTok{  }\KeywordTok{pivot_longer}\NormalTok{(}\DataTypeTok{names_to =} \StringTok{"year"}\NormalTok{, }\DataTypeTok{values_to =} \StringTok{"n"}\NormalTok{, }\DataTypeTok{cols =} \DecValTok{2}\OperatorTok{:}\DecValTok{4}\NormalTok{) }\OperatorTok\StringTok{ }
\StringTok{  }\KeywordTok{head}\NormalTok{(}\DecValTok{5}\NormalTok{)}
\end{Highlighting}
\end{Shaded}

\begin{verbatim}
## # A tibble: 5 x 3
##   country year      n
##   <chr>   <chr> <dbl>
## 1 FR      2011   7000
## 2 FR      2012   6900
## 3 FR      2013   7000
## 4 DE      2011   5800
## 5 DE      2012   6000
\end{verbatim}

\begin{itemize}
\tightlist
\item
  Here the columns 2:4 are transposed into a \texttt{year} column.
\item
  We put the corresponding count values into a column called \texttt{n}.
\end{itemize}

\hypertarget{other-approaches-to-do-the-same-thing.}{%
\subsection{Other approaches to do the same
thing.}\label{other-approaches-to-do-the-same-thing.}}

\begin{Shaded}
\begin{Highlighting}[]
\CommentTok{# Method 2}
\NormalTok{EDAWR}\OperatorTok{::}\NormalTok{cases }\OperatorTok\StringTok{ }
\StringTok{  }\KeywordTok{pivot_longer}\NormalTok{(}\DataTypeTok{names_to =} \StringTok{"year"}\NormalTok{, }\DataTypeTok{values_to =} \StringTok{"n"}\NormalTok{, }\OperatorTok{-}\NormalTok{country) }\OperatorTok\StringTok{ }
\StringTok{  }\KeywordTok{head}\NormalTok{(}\DecValTok{5}\NormalTok{)}
\end{Highlighting}
\end{Shaded}

\begin{verbatim}
## # A tibble: 5 x 3
##   country year      n
##   <chr>   <chr> <dbl>
## 1 FR      2011   7000
## 2 FR      2012   6900
## 3 FR      2013   7000
## 4 DE      2011   5800
## 5 DE      2012   6000
\end{verbatim}

\begin{Shaded}
\begin{Highlighting}[]
\CommentTok{# Method 3}
\CommentTok{# EDAWR::cases %>% }
\CommentTok{#   pivot_longer(names_to = "year", values_to = "n", c(`2011`, `2012`, `2013`)) }
\CommentTok{# Method 4}
\CommentTok{# EDAWR::cases %>% }
\CommentTok{#   pivot_longer(names_to = "year", values_to = "n",  `2011`:`2013`) }
\end{Highlighting}
\end{Shaded}

\hypertarget{pivot_wider-makes-longer-data-formats-wider.}{%
\section{\texorpdfstring{\texttt{pivot\_wider()}: Makes longer data
formats
wider.}{pivot\_wider(): Makes longer data formats wider.}}\label{pivot_wider-makes-longer-data-formats-wider.}}

\begin{Shaded}
\begin{Highlighting}[]
\NormalTok{EDAWR}\OperatorTok{::}\NormalTok{pollution }\OperatorTok\StringTok{ }
\StringTok{  }\KeywordTok{head}\NormalTok{(}\DecValTok{5}\NormalTok{)}
\end{Highlighting}
\end{Shaded}

\begin{verbatim}
##       city  size amount
## 1 New York large     23
## 2 New York small     14
## 3   London large     22
## 4   London small     16
## 5  Beijing large    121
\end{verbatim}

\begin{Shaded}
\begin{Highlighting}[]
\NormalTok{EDAWR}\OperatorTok{::}\NormalTok{pollution }\OperatorTok\StringTok{ }
\StringTok{  }\KeywordTok{pivot_wider}\NormalTok{(}\DataTypeTok{names_from =} \StringTok{"size"}\NormalTok{, }
              \DataTypeTok{values_from =} \StringTok{"amount"}\NormalTok{)}
\end{Highlighting}
\end{Shaded}

\begin{verbatim}
## # A tibble: 3 x 3
##   city     large small
##   <chr>    <dbl> <dbl>
## 1 New York    23    14
## 2 London      22    16
## 3 Beijing    121    56
\end{verbatim}

\hypertarget{when-could-i-use-these-operations}{%
\section{When could I use these
operations?}\label{when-could-i-use-these-operations}}

\begin{itemize}
\tightlist
\item
  Data visualization with \texttt{ggplot2}.
\end{itemize}

\textbf{Read}:
\url{https://tidyr.tidyverse.org/articles/pivot.html\#manual-specs}

\hypertarget{separate}{%
\subsection{\texorpdfstring{\texttt{separate()}}{separate()}}\label{separate}}

To separate a character column into multiple columns using a regular
expression separator.

The following code seperates date into multiple columns. ``-'' is used
to seperate between words.

\begin{Shaded}
\begin{Highlighting}[]
\NormalTok{EDAWR}\OperatorTok{::}\NormalTok{storms }\OperatorTok\StringTok{ }
\StringTok{  }\KeywordTok{head}\NormalTok{(}\DecValTok{3}\NormalTok{)}
\end{Highlighting}
\end{Shaded}

\begin{verbatim}
##     storm wind pressure       date
## 1 Alberto  110     1007 2000-08-03
## 2    Alex   45     1009 1998-07-27
## 3 Allison   65     1005 1995-06-03
\end{verbatim}

\begin{Shaded}
\begin{Highlighting}[]
\NormalTok{storms2 <-}\StringTok{ }\NormalTok{EDAWR}\OperatorTok{::}\NormalTok{storms }\OperatorTok\StringTok{ }
\StringTok{  }\KeywordTok{separate}\NormalTok{(date, }\KeywordTok{c}\NormalTok{(}\StringTok{"y"}\NormalTok{, }\StringTok{"m"}\NormalTok{, }\StringTok{"d"}\NormalTok{), }\DataTypeTok{sep=}\StringTok{"-"}\NormalTok{) }\CommentTok{# sep = "-"}
\NormalTok{storms2}
\end{Highlighting}
\end{Shaded}

\begin{verbatim}
## # A tibble: 6 x 6
##   storm    wind pressure y     m     d    
##   <chr>   <int>    <int> <chr> <chr> <chr>
## 1 Alberto   110     1007 2000  08    03   
## 2 Alex       45     1009 1998  07    27   
## 3 Allison    65     1005 1995  06    03   
## 4 Ana        40     1013 1997  06    30   
## 5 Arlene     50     1010 1999  06    11   
## 6 Arthur     45     1010 1996  06    17
\end{verbatim}

\hypertarget{unite}{%
\section{\texorpdfstring{\texttt{unite()}}{unite()}}\label{unite}}

Paste together multiple columns into one.

The following code combines \texttt{y}, \texttt{m} and \texttt{d} in
\texttt{storm2} using ``-''.

\begin{Shaded}
\begin{Highlighting}[]
\NormalTok{storms2 }\OperatorTok
\StringTok{  }\KeywordTok{unite}\NormalTok{(date, y, m, d, }\DataTypeTok{sep =} \StringTok{"-"}\NormalTok{)}
\end{Highlighting}
\end{Shaded}

\begin{verbatim}
## # A tibble: 6 x 4
##   storm    wind pressure date      
##   <chr>   <int>    <int> <chr>     
## 1 Alberto   110     1007 2000-08-03
## 2 Alex       45     1009 1998-07-27
## 3 Allison    65     1005 1995-06-03
## 4 Ana        40     1013 1997-06-30
## 5 Arlene     50     1010 1999-06-11
## 6 Arthur     45     1010 1996-06-17
\end{verbatim}

Note that \texttt{unite()} and \texttt{separate()} are inverse
operations.

\hypertarget{dplyr-package}{%
\section{\texorpdfstring{\texttt{dplyr}
package}{dplyr package}}\label{dplyr-package}}

\hypertarget{group_by}{%
\section{\texorpdfstring{\texttt{group\_by()}}{group\_by()}}\label{group_by}}

To define a grouping of rows based on a column:

\begin{Shaded}
\begin{Highlighting}[]
\NormalTok{iris }\OperatorTok\StringTok{ }
\StringTok{  }\KeywordTok{group_by}\NormalTok{(Species) }\OperatorTok
\StringTok{  }\KeywordTok{head}\NormalTok{(}\DecValTok{4}\NormalTok{)}
\end{Highlighting}
\end{Shaded}

\begin{verbatim}
## # A tibble: 4 x 5
## # Groups:   Species [1]
##   Sepal.Length Sepal.Width Petal.Length Petal.Width Species
##          <dbl>       <dbl>        <dbl>       <dbl> <fct>  
## 1          5.1         3.5          1.4         0.2 setosa 
## 2          4.9         3            1.4         0.2 setosa 
## 3          4.7         3.2          1.3         0.2 setosa 
## 4          4.6         3.1          1.5         0.2 setosa
\end{verbatim}

\begin{Shaded}
\begin{Highlighting}[]
\NormalTok{iris }\OperatorTok\StringTok{ }
\StringTok{  }\KeywordTok{group_by}\NormalTok{(Species) }\OperatorTok
\StringTok{  }\KeywordTok{head}\NormalTok{(}\DecValTok{4}\NormalTok{) }\OperatorTok\StringTok{ }\NormalTok{class}
\end{Highlighting}
\end{Shaded}

\begin{verbatim}
## [1] "grouped_df" "tbl_df"     "tbl"        "data.frame"
\end{verbatim}

\begin{itemize}
\item
  This doesn't actually change anything in the output.
\item
  The only difference is that when it prints, we're told about the
  groups.
\item
  But it will play a big role in how other \texttt{dplyr} functions
  work.
\end{itemize}

\hypertarget{summarize-american-or-summarise-british}{%
\section{\texorpdfstring{\texttt{summarize()} (American) or
\texttt{summarise()}
(British)}{summarize() (American) or summarise() (British)}}\label{summarize-american-or-summarise-british}}

\texttt{summarize()} or \texttt{summarise()} in \texttt{dplyr} gives you
single numerical summaries.

\begin{Shaded}
\begin{Highlighting}[]
\CommentTok{# Ungrouped}
\NormalTok{iris }\OperatorTok\StringTok{ }
\StringTok{  }\KeywordTok{summarize}\NormalTok{(}\DataTypeTok{Sepal.Length =} \KeywordTok{mean}\NormalTok{(Sepal.Length),}
            \DataTypeTok{Sepal.Width =} \KeywordTok{mean}\NormalTok{(Sepal.Width))}
\end{Highlighting}
\end{Shaded}

\begin{verbatim}
##   Sepal.Length Sepal.Width
## 1     5.843333    3.057333
\end{verbatim}

\begin{Shaded}
\begin{Highlighting}[]
\CommentTok{# Grouped by number of Species}
\NormalTok{iris }\OperatorTok
\StringTok{  }\KeywordTok{group_by}\NormalTok{(Species) }\OperatorTok
\StringTok{  }\KeywordTok{summarize}\NormalTok{(}\DataTypeTok{Sepal.Length =} \KeywordTok{mean}\NormalTok{(Sepal.Length),}
            \DataTypeTok{Sepal.Width =} \KeywordTok{mean}\NormalTok{(Sepal.Width))}
\end{Highlighting}
\end{Shaded}

\begin{verbatim}
## # A tibble: 3 x 3
##   Species    Sepal.Length Sepal.Width
##   <fct>             <dbl>       <dbl>
## 1 setosa             5.01        3.43
## 2 versicolor         5.94        2.77
## 3 virginica          6.59        2.97
\end{verbatim}

\begin{center}\rule{0.5\linewidth}{0.5pt}\end{center}

\begin{Shaded}
\begin{Highlighting}[]
\NormalTok{iris }\OperatorTok
\StringTok{  }\KeywordTok{group_by}\NormalTok{(Species) }\OperatorTok
\StringTok{  }\KeywordTok{summarize}\NormalTok{(}\DataTypeTok{Sepal.Width_mean =} \KeywordTok{mean}\NormalTok{(Sepal.Width),}
            \DataTypeTok{Sepal.Width_max =} \KeywordTok{max}\NormalTok{(Sepal.Width),}
            \DataTypeTok{Sepal.Length_mean =} \KeywordTok{mean}\NormalTok{(Sepal.Length),}
            \DataTypeTok{Sepal.Length_max =} \KeywordTok{max}\NormalTok{(Sepal.Length))}
\end{Highlighting}
\end{Shaded}

\begin{verbatim}
## # A tibble: 3 x 5
##   Species    Sepal.Width_mean Sepal.Width_max Sepal.Length_mean Sepal.Length_max
##   <fct>                 <dbl>           <dbl>             <dbl>            <dbl>
## 1 setosa                 3.43             4.4              5.01              5.8
## 2 versicolor             2.77             3.4              5.94              7  
## 3 virginica              2.97             3.8              6.59              7.9
\end{verbatim}

\hypertarget{ungroup}{%
\section{\texorpdfstring{\texttt{ungroup()}}{ungroup()}}\label{ungroup}}

To remove groupings structure from a data frame or a tibble.

\begin{Shaded}
\begin{Highlighting}[]
\NormalTok{iris }\OperatorTok
\StringTok{  }\KeywordTok{group_by}\NormalTok{(Species) }\OperatorTok
\StringTok{  }\KeywordTok{ungroup}\NormalTok{() }\OperatorTok
\StringTok{  }\KeywordTok{summarize}\NormalTok{(}\DataTypeTok{Sepal.Width =} \KeywordTok{mean}\NormalTok{(Sepal.Width),}
            \DataTypeTok{Petal.Width =} \KeywordTok{mean}\NormalTok{(Petal.Width))}
\end{Highlighting}
\end{Shaded}

\begin{verbatim}
## # A tibble: 1 x 2
##   Sepal.Width Petal.Width
##         <dbl>       <dbl>
## 1        3.06        1.20
\end{verbatim}

\hypertarget{join-operations}{%
\section{Join operations}\label{join-operations}}

A ``join'' operation combines two data sets. There are 4 types of join
operations.

\begin{itemize}
\tightlist
\item
  \textbf{Inner join} (or just \textbf{join}): keeps just the rows each
  table that match the condition.
\item
  \textbf{Left outer join} (or just \textbf{left join}): keeps all rows
  in the first table, and just the rows in the second table that match
  the condition.
\item
  \textbf{Right outer join} (or just \textbf{right join}): keeps just
  the rows in the first table that match the condition, and all rows in
  the second table.
\item
  \textbf{Full outer join} (or just \textbf{full join}): keeps all rows
  in both tables.
\end{itemize}

Note Column values that cannot be filled in are assigned \texttt{NA}
values.

\hypertarget{illustration-with-two-simple-data-sets.}{%
\section{Illustration with two simple data
sets.}\label{illustration-with-two-simple-data-sets.}}

\begin{Shaded}
\begin{Highlighting}[]
\NormalTok{tab1_age <-}\StringTok{ }\KeywordTok{data.frame}\NormalTok{(}\DataTypeTok{name =} \KeywordTok{c}\NormalTok{(}\StringTok{"Ann"}\NormalTok{, }\StringTok{"Jenny"}\NormalTok{, }\StringTok{"Andrew"}\NormalTok{), }
                  \DataTypeTok{age =} \KeywordTok{c}\NormalTok{(}\DecValTok{70}\NormalTok{, }\DecValTok{52}\NormalTok{, }\DecValTok{40}\NormalTok{),}
                  \DataTypeTok{stringsAsFactors =} \OtherTok{FALSE}\NormalTok{)}
\NormalTok{tab2_testresult <-}\StringTok{ }\KeywordTok{data.frame}\NormalTok{(}\DataTypeTok{name =} \KeywordTok{c}\NormalTok{(}\StringTok{"Ann"}\NormalTok{, }\StringTok{"Nick"}\NormalTok{, }\StringTok{"Anderw"}\NormalTok{),}
                  \DataTypeTok{result =} \KeywordTok{c}\NormalTok{(}\StringTok{"negative"}\NormalTok{, }\StringTok{"positive"}\NormalTok{, }\StringTok{"negative"}\NormalTok{),}
                  \DataTypeTok{stringsAsFactors =} \OtherTok{FALSE}\NormalTok{)}
\NormalTok{tab1_age}
\end{Highlighting}
\end{Shaded}

\begin{verbatim}
##     name age
## 1    Ann  70
## 2  Jenny  52
## 3 Andrew  40
\end{verbatim}

\begin{Shaded}
\begin{Highlighting}[]
\NormalTok{tab2_testresult}
\end{Highlighting}
\end{Shaded}

\begin{verbatim}
##     name   result
## 1    Ann negative
## 2   Nick positive
## 3 Anderw negative
\end{verbatim}

\hypertarget{inner_join}{%
\section{\texorpdfstring{\texttt{inner\_join()}}{inner\_join()}}\label{inner_join}}

name column is common to both \texttt{tab1\_age} and
\texttt{tab2\_testresult}. This keeps only the common rows
(intersection) in both datasets.

\begin{Shaded}
\begin{Highlighting}[]
\KeywordTok{inner_join}\NormalTok{(}\DataTypeTok{x =}\NormalTok{ tab1_age, }\DataTypeTok{y =}\NormalTok{ tab2_testresult, }\DataTypeTok{by =} \StringTok{"name"}\NormalTok{)}
\end{Highlighting}
\end{Shaded}

\begin{verbatim}
##   name age   result
## 1  Ann  70 negative
\end{verbatim}

\hypertarget{left_join}{%
\section{\texorpdfstring{\texttt{left\_join()}}{left\_join()}}\label{left_join}}

This keeps all names from \texttt{tab1\_age}.

\begin{Shaded}
\begin{Highlighting}[]
\KeywordTok{left_join}\NormalTok{(}\DataTypeTok{x =}\NormalTok{ tab1_age, }\DataTypeTok{y =}\NormalTok{ tab2_testresult, }\DataTypeTok{by =} \KeywordTok{c}\NormalTok{(}\StringTok{"name"}\NormalTok{ =}\StringTok{ "name"}\NormalTok{))}
\end{Highlighting}
\end{Shaded}

\begin{verbatim}
##     name age   result
## 1    Ann  70 negative
## 2  Jenny  52     <NA>
## 3 Andrew  40     <NA>
\end{verbatim}

\hypertarget{right_join}{%
\section{\texorpdfstring{\texttt{right\_join()}}{right\_join()}}\label{right_join}}

This keeps all names from \texttt{tab2\_testresult}.

\begin{Shaded}
\begin{Highlighting}[]
\KeywordTok{right_join}\NormalTok{(}\DataTypeTok{x =}\NormalTok{ tab1_age, }\DataTypeTok{y =}\NormalTok{ tab2_testresult, }\DataTypeTok{by =} \StringTok{"name"}\NormalTok{)}
\end{Highlighting}
\end{Shaded}

\begin{verbatim}
##     name age   result
## 1    Ann  70 negative
## 2   Nick  NA positive
## 3 Anderw  NA negative
\end{verbatim}

\hypertarget{full_join}{%
\section{\texorpdfstring{\texttt{full\_join()}}{full\_join()}}\label{full_join}}

This keeps all rows from both data frames.

\begin{Shaded}
\begin{Highlighting}[]
\KeywordTok{full_join}\NormalTok{(}\DataTypeTok{x =}\NormalTok{ tab1_age, }\DataTypeTok{y =}\NormalTok{ tab2_testresult, }\DataTypeTok{by =} \StringTok{"name"}\NormalTok{)}
\end{Highlighting}
\end{Shaded}

\begin{verbatim}
##     name age   result
## 1    Ann  70 negative
## 2  Jenny  52     <NA>
## 3 Andrew  40     <NA>
## 4   Nick  NA positive
## 5 Anderw  NA negative
\end{verbatim}

\hypertarget{summary}{%
\section{Summary}\label{summary}}

\begin{itemize}
\tightlist
\item
  \texttt{tidyr} is a package for manipulating the structure of data
  frames
\item
  \texttt{pivot\_longer()}: make wide data longer
\item
  \texttt{pivot\_wider()}: make long data wider
\item
  \texttt{unite()} and \texttt{separate()}: combine or split columns
\item
  \texttt{dplyr} has advanced functionality that mirrors SQL
\item
  \texttt{group\_by()}: create groups of rows according to a condition
\item
  \texttt{summarize()}: apply computations across groups of rows
\item
  \texttt{*\_join()} where \texttt{*} = \texttt{inner}, \texttt{left},
  \texttt{right}, or \texttt{full}: join two data frames together
  according to common values in certain columns, and \texttt{*}
  indicates how many rows to keep.
\end{itemize}

\end{document}
